\documentclass[TS,authoryear,toc,lsstdraft]{lsstdoc}
% lsstdoc documentation: https://lsst-texmf.lsst.io/lsstdoc.html

% Package imports go here.

% Local commands go here.

\input meta.tex

% To add a short-form title:
\title{T\&S Software Products}

% Optional subtitle
% \setDocSubtitle{A subtitle}

\author{%
Gabriele Comoretto
}

\setDocRef{TSTN-001}

\date{\today}

% Optional: name of the document's curator
% \setDocCurator{The Curator of this Document}

\setDocAbstract{%
Telescope \& Site Software Products
}

% Change history defined here.
% Order: oldest first.
% Fields: VERSION, DATE, DESCRIPTION, OWNER NAME.
% See LPM-51 for version number policy.
\setDocChangeRecord{%
  \addtohist{}{2019-06-18}{draft}{Gabriele Comoretto}
}


\begin{document}

\setDocUpstreamLocation{\url{https://github.com/ts-e/tstn-001}}
\setDocUpstreamVersion{\vcsrevision}

\maketitle

\section{Introduction}

This document should document all T\&S software products.


\subsection{Applicable Documents}

%\begin{tabbing}
%AUTH-NUM\= \kill
%\citeds{LDM-148} \>     DM Architecture\\
%\end{tabbing}

The \citeds{TsDevGuide} is applicable to this document.


\newpage
\section{T\&S Software Products} \label{sec:tsswp}


\subsection{Non CSC Software Products} \label{sec:nonops}



\subsubsection{SAL}

\textbf{Acronym}: 

\textbf{Description}:

\begin{longtable}[]{p{2cm}p{2cm}p{2cm}p{2cm}p{2cm}p{2cm}}
\hline
\multicolumn{4}{c}{\textbf{Personell}} \\ \hline
SW Prod. Owner & Developers & LSST PoC   & Product Owner & Provider Company & External PoC \\ \hline
               &            &            &               &                  &              \\ \hline
\end{longtable}

\begin{longtable}[]{rll}
\hline
\multicolumn{3}{c}{\textbf{Resources}} \\ \hline
src repository    & \url{https://github.com/lsst-ts/ts_sal} & \textit{python} \\ \hline
binary pkg repo   & \textit{specify where the binary packages are stored (Nexus, Anaconda, others)} & \textit{pypi}\\ \hline
\hline
jenkins job    & & \\ \hline
\end{longtable}

\begin{longtable}[]{rll}
\hline
\multicolumn{3}{c}{\textbf{Documentation}} \\ \hline
lsst.io & NA & \\ \hline
Requirements & NA & \\ \hline
Design Doc & NA & \\ \hline
User Manual & NA & \\ \hline
Test Spec & NA & \\ \hline
\end{longtable}


\subsubsection{SALOBJ}

\textbf{Acronym}: 

\textbf{Description}:

\begin{longtable}[]{p{2cm}p{2cm}p{2cm}p{2cm}p{2cm}p{2cm}}
\hline
\multicolumn{4}{c}{\textbf{Personell}} \\ \hline
SW Prod. Owner & Developers & LSST PoC   & Product Owner & Provider Company & External PoC \\ \hline
               &            &            &               &                  &              \\ \hline
\end{longtable}

\begin{longtable}[]{rll}
\hline
\multicolumn{3}{c}{\textbf{Resources}} \\ \hline
src repository & \url{https://github.com/lsst-ts/ts_salobj} & python \\ \hline
binary pkg repo   & \textit{specify where the binary packages are stored (Nexus, Anaconda, others)} & \textit{pypi}\\ \hline
\hline
jenkins job    & & \\ \hline
\end{longtable}

\begin{longtable}[]{rll}
\hline
\multicolumn{3}{c}{\textbf{Documentation}} \\ \hline
lsst.io & NA & \\ \hline
Requirements & NA & \\ \hline
Design Doc & NA & \\ \hline
User Manual & NA & \\ \hline
Test Spec & NA & \\ \hline
\end{longtable}




\subsection{CSCs - XML Interfaces} \label{sec:cscs}


\subsubsection{Generic XML Interface}

\textbf{Acronym}: 

\textbf{Description}:
This is the top level XML configuration file, used by all others XML interfaces, and therefore by all others CSCs.

\begin{longtable}[]{p{2cm}p{2cm}p{2cm}p{2cm}p{2cm}p{2cm}}
\hline
\multicolumn{4}{c}{\textbf{Personell}} \\ \hline
Owner(s)     & Developers & LSST PoC   & Product Owner & Provider Company & External PoC \\ \hline
             & All        &            &               &                  &              \\ \hline
\end{longtable}

\begin{longtable}[]{rll}
\hline
\multicolumn{3}{c}{\textbf{Resources}} \\ \hline
xml repository & \url{https://github.com/lsst-ts/ts-xml/ATAOTS} & xml \\ \hline
xml pkgs repo  &                                                &  \\ \hline
\end{longtable}


\subsubsection{ATAOS}

\textbf{Acronym}: 

\textbf{Description}:

\begin{longtable}[]{p{2cm}p{2cm}p{2cm}p{2cm}p{2cm}p{2cm}}
\hline
\multicolumn{4}{c}{\textbf{Personell}} \\ \hline
CSC Owner(s) & Developers & LSST PoC   & Product Owner & Provider Company & External PoC \\ \hline
T.Riberiro   & T.Riberiro & P.Ingraham & P.Ingraham  & &  \\ \hline
\end{longtable}

\begin{longtable}[]{rll}
\hline
\multicolumn{3}{c}{\textbf{Resources}} \\ \hline
xml repository & \url{https://github.com/lsst-ts/ts-xml/ATAOTS} & xml \\ \hline
xml pkgs repo  & \textit{specify where the xmls pkgs are stored (Nexus, anaconda, other)} & \textit{rpm} \\ \hline
csc repository & \url{https://github.com/lsst-ts/ts_ataos} & python \\ \hline
csc pkg repo   & \textit{specify where the csc packages are stored (Nexus, Anaconda, others)} & \textit{pypi}\\ \hline
\hline
jenkins job    & & \\ \hline
\end{longtable}

\begin{longtable}[]{rll}
\hline
\multicolumn{3}{c}{\textbf{Documentation}} \\ \hline
lsst.io & NA & \\ \hline
Requirements & NA & \\ \hline
Design Doc & NA & \\ \hline
User Manual & NA & \\ \hline
Test Spec & NA & \\ \hline
\end{longtable}



\newpage
\section{T\&S 3rd Party Software} \label{sec:3rd}

\subsection{DDS OpenSplice}

\textbf{Acronym}:

\textbf{Description}:

\begin{longtable}[]{cccc}
\hline
\multicolumn{4}{c}{\textbf{Personell}} \\ \hline
LSST PoC   & Product Owner & Provider Company & External PoC \\ \hline
           &               &                  &              \\ \hline
\end{longtable}

\begin{longtable}[]{rll}
\hline
\multicolumn{3}{c}{\textbf{Resources}} \\ \hline
Source Code Repository     &  &  \\ \hline
Binary Package Repository  &  & \\ \hline
Documentation              &  &  \\ \hline
\end{longtable}




\appendix
\newpage
\section{References} \label{sec:bib}
\bibliography{lsst,refs_ads,refs,books,local}


%Make sure lsst-texmf/bin/generateAcronyms.py is in your path
\section{Acronyms used in this document}\label{sec:acronyms}
\addtocounter{table}{-1}
\begin{longtable}{|l|p{0.8\textwidth}|}\hline
\textbf{Acronym} & \textbf{Description}  \\\hline

LSST & Large Synoptic Survey Telescope \\\hline
T\&S & Telescope and Site \\\hline
TS & Test Specification \\\hline
XML & eXtensible Markup Language \\\hline
\end{longtable}

\end{document}
